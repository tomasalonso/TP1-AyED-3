\section{Introducción}

El trabajo realizado implementa la simulacion de flujo de fluidos basada en las
ecuaciones de Navier-Stokes, la eleccion de utilizar estas ecuaciones es porque
se consigue un resultado que no es numericamente preciso pero es lo
suficientemente estable, simple y fiel a la realidad como para dejar ver lo que
este trabajo intenta observar.



El objetivo final es poder observar las
diferencias de rendimiento de programas en codigo en C comparados con la misma
logica pero implementada en Assembler utilizando el set de instrucciones SSE
propio de los procesadores intel.



En principio se explican las decisiones de implementacion del codigo, ya sean de
comodidad y simpleza o por mejoria de rendimiento, asi como tambien se habla
sobre intentos fallidos por intentar mejorar la performance al utilizar
extensivamente la metodologia SIMD, pero encontrandonos con inconvenientes
inesperados. Tambien se puede observar que es lo que efectivamente realiza el
codigo y como esto influyó en las decisiones tomadas.



Una vez en claro como funciona el programa y porque se decidio hacerlo asi, se
muestra la experimentacion exhaustiva que se hizo para medir las diferencias de
tiempos entre todo el trabajo con las funciones en C en comparacion con todas
las funciones nuevas en assembler, esto en relacion con el rendimiento comparado
entre cada funcion en C con su contraparte en assembler. Con todo esto se puede
ver claramente que implementacion es mas rapida asi como distinguir que
funciones son las que hacen la mayor diferencia en general.
