\section{Conclusión}

En concuclusion podemos decir que este trabajo nos permitio aprender y volcarnos en muchos aspectos distintos,
por ejemplo la programacion con SIMD y el procesamiento de matrices en assembler asi como en el desarrollo
de un informe profundo de experimentacion y analisis.


Se puede ver claramente lo que se buscaba sobre performance del programa
, ademas de poder justificarlo con pruebas y experimentos,
efectivamente se nota la mejora que causa el uso de la metodologia SIMD y assembler cuando se trata de
implementaciones que necesitan el procesamiento sistematico de muchos datos con caracteristicas iguales.


Tambien descubrimos con la implementacion de lin_solve (lin_solve_largo)
que no siempre más paralelizacion y mas trabajo por ciclo lleva a una mejoria temporal
sino que es mas importente aprovechar bien la logica de la memoria cache y no realizar
saltos lejanos en memoria en cada iteracion, como se penso en esta funcion en un principio.
Otro problema que encontramos fue que no se puedo sacar una conclusion clara de las tablas de diferencias
ni sobre por que los tests presentan el comportamiento particular mostrado en los resultados.
Quizas lo necesario seria mas conocimiento profundo sobre como trabaja el programa en general.


Cabe aclarar que este trabajo tambien sirvio como primer informe en la carrera de 2 integrantes del grupo, lo cual
aporto para lograr que sea una experiencia en la que se aprendio mucho sobre como se arman, revisan
y se desarrollan informes de investigacion.
